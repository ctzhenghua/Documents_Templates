\documentclass[UTF8,a4paper,8pt]{ctexbook} 

 \usepackage{graphicx}%学习插入图
 \usepackage{verbatim}%学习注释多行
 \usepackage{booktabs}%表格
 \usepackage{geometry}%图片
 \usepackage{amsmath} 
 \usepackage{amssymb}
 \usepackage{listings}%代码
 \usepackage{xcolor}  %颜色
 \usepackage{enumitem}%列表格式
 \CTEXsetup[format+={\flushleft}]{section}


\geometry{left=1.6cm,right=1.8cm,top=2cm,bottom=1.7cm} %设置文章宽度

\pagestyle{plain} 		  %设置页面布局
\author{郑华}
\title{聊天技巧总结}
 %代码效果定义
 
 \definecolor{codegreen}{rgb}{0,0.6,0}
 \definecolor{codegray}{rgb}{0.5,0.5,0.5}
 \definecolor{codepurple}{rgb}{0.58,0,0.82}
 \definecolor{backcolour}{rgb}{0.95,0.95,0.92}
 
 \lstdefinestyle{mystyle}{
 	language = {c++},%lua 语言指定方法
 	backgroundcolor=\color{backcolour},   
 	commentstyle=\color{codegreen},
 	keywordstyle=\color{magenta},
 	numberstyle=\tiny\color{codegray},
 	stringstyle=\color{codepurple},
 	basicstyle=\footnotesize,
 	breakatwhitespace=false,         
 	breaklines=true,                 
 	captionpos=b,                    
 	keepspaces=true,                 
 	%numbers=left,                    
 	%numbersep=5pt,                  
 	showspaces=false,                
 	showstringspaces=false,
 	showtabs=false,                  
 	tabsize=2
 }
\lstset{style=mystyle, escapeinside=``}

\begin{document}          %正文排版开始
 	\maketitle
 \chapter{自我经验篇}
 \section{共同爱好往往是打开话题的敲门砖}
	 \subsection{找相同点}
		 \paragraph{1.读书}
			 \begin{itemize}
			 	\item 围城
			 	\item 追风筝的人
			 	\item 乖,摸摸头
			 	\item 从你的全世界路过
			 	\item 挪威的森林
			 	\item 了不起的盖茨比
			 	\item 撒哈拉的故事
			 	\item 岛上书店
			 	\item 茶花女
			 	\item 爵迹
			 	\item 一个人的朝圣
			 	\item 我不喜欢这个世界,我只喜欢你
			 	\item 穷爸爸富爸爸
			 	\item 走错路,也能到对岸
			 	\item 因为痛,所以叫青春
			 	\item 不曾走过,怎会懂得
			 	\item 跟任何人都能聊的来
			 \end{itemize}
		 \paragraph{2.吉他}
			 \begin{itemize}
				 \item 练习-指头
				 \item 曲子
			 \end{itemize}
		 \paragraph{3.篮球-锻炼}
		 \paragraph{4.吃}
			 \subparagraph{类型}
				 \begin{itemize}
				 	\item 烧烤
				 	\item 牛排
				 	\item 火锅
				 	\item 经典吃饭地名
				 \end{itemize}
			 \subparagraph{口味}
		 \paragraph{5.逛-宅}
			 \begin{itemize}
			 	\item 南京
			 	\item 深圳
			 	\item 马嵬驿
			 	\item 袁家村
			 \end{itemize}
		 \paragraph{6.收藏}
			 \begin{itemize}
				 \item 鞋
				 \item 吊牌
			 \end{itemize}
		 \paragraph{7.唱歌}
			 \begin{itemize}
			 	\item 歌曲类型
			 	\item 歌手
			 \end{itemize}
		 \paragraph{8.游戏}
			 \begin{itemize}
			 	\item  LOL
			 	\item  Dota2
			 	\item  生死狙击
			 	\item  CF
			 	\item  坦克世界
			 	\item  小游戏s
			 \end{itemize}
		 \paragraph{9.经历}
			 \begin{itemize}
			 	\item 大学
			 	\item 高中
			 	\item 恋爱
			 	\item 职业
			 	\item 旅行
			 	\item 家庭
			 \end{itemize}
		 \paragraph{*八卦}
			 \begin{itemize}
			 	\item  让我听听你..的故事吧
			 \end{itemize}
	 \subsection{主动约会出去实践爱好}
 \section{沟通技巧}
	\subsection{性格}
		刚开始一定要说清楚,要不越往后越不好处理,耿直、委婉、
	\subsection{幽默} 

 \section{段子}
	 \begin{itemize}
	 	\item 
	 \end{itemize} 
	 	 
 \section{撩妹技巧}

\chapter{读书学习篇}
	\section{<5分钟和陌生人成为朋友>}
		\subsection{主动-我很友好,如果你也愿意的话,我愿意与你交流}
			It takes a certain amount of risk to begin a conversation with a stranger. Most shy people don't start conversations because they fear being rejected. Of course,this prevents them from reaching out to others. Remember that risk taking and rejection are part of life,and to be overly sensitive is conterproductive.\textbf{And anyway, what's so bad about being rejected by someone you don't even Konw?}
			
			Rejections are a part of everyday life. \textbf{Don't let them keep you from reaching out to others.}
			
		\subsection{确立信任感有方法}
			\subparagraph{第1层次:套话}礼仪性问候,这些提供了一个很好的披露自由信息的机会。低层次的信息披露告诉他人你的态度很开放和友善,如果环境允许的话,你可以和他继续对话。
			
			\subparagraph{第2层次:交换基本个人信息}告诉你的工作、从哪来、喜欢干什么、或者其他一些你当前参加的项目或活动。这层对话向对方提供了可以对比和挖掘的经历和信息背景。在这一点上人们开始相互了解对方。
			
			\subparagraph{第3层次:表现各个事物上的个人观点和偏好}你可以表达你的观点、价值观和关注事物。你可以告诉他人你对周围世界诚实的想法和感觉,以开放式的方式表达自己的观点,鼓励他人分享他们对各种话题的想法。 请记住,\textbf{人们总会有不同的观点。}
			
			\textbf{一段好的对话不应当是一场争辩,不存在胜者和输家,而是观点和想法的交换。一场思想开明的讨论,不是争辩,而是在保持一段对话的同时让对话的参与者 对彼此有更深层次的了解。}
			
			
			\subparagraph{第4层次:个人感觉}告知对方你是什么样的人,你是谁,你在意什么。当你表达个人-期望、目标、理想、喜好、快乐和悲伤时,人们会表示认同。因为我们都能彼此分享这些基本的情感体验。
			
			\textbf{当你披露你的情感和观点时,请记住说“我觉得,我认为,我相信”等。}
			
			\subparagraph{注意}:
				\begin{itemize}
					\item 不要走向信息披露的另一个极端-什么都说: 这会让对方很烦
					\item 对自己实事求是,如果你夸大自己好的方面,而掩盖自己错误的话,人们会渐渐意识到你讲的不是一个真实的故事。
					\item 展示你的目标:你会惊喜的发现大多数人表示共鸣。
					\item 让别人了解你:不要担心让别人感到厌倦,大多数对交新朋友很感兴趣,如果相互之间感兴趣的话,让对方知道你是谁很重要。\textbf{你不需要取悦你见到的人,但是应该表现得乐观向上。}大多数人很珍惜人际关系。
				\end{itemize}
			
	\section{<人际交往秘诀>}
	
	\section{<怎么说别人才会听你的>}
	
	\section{<50种改善沟通的1分钟窍门>}
	
	\section{<别独自用餐>}
	
	\section{<人性的弱点>}
	
	\section{<果冻效应:让你一语中的>}
	
	\section{<博恩-崔西 口才经:如何在任何场合说服任何人>}
	
	\section{<如何与难相处的人打交道>}
	
	\section{<人见人爱96计:成功社会秘籍>}
	
	\section{<知道听众想要什么>}
	
	\section{<怎么销售你自己>}
	
	\section{<高难度谈话>}
	
	\section{<FBI教你读心术>}
	
	\section{<谈判力>}
	
	\section{<时间管理>}
	
	\section{<如何让你爱的人爱上你>}
	
	\section{<身体语言密码>}
	
	\section{<读人>}
	
	\section{<卓有成效的管理者>}
	
	\section{<90秒建立职场人脉>}
	
	\section{<职场生存指南>}
	
	\section{<刚柔并济领导力>}
	
	\section{<发展你的职业性格:MBTI帮助你发挥自身潜力>}
	
	\section{<影响力>}
	
	\section{<人间游戏:人际关系心理学>}
		    
\end{document} 
 		    