\documentclass[UTF8,a4paper,8pt]{ctexart} 

\usepackage{graphicx}%学习插入图
\usepackage{verbatim}%学习注释多行
\usepackage{booktabs}%表格
\usepackage{geometry}%图片
\usepackage{amsmath} 
\usepackage{amssymb}
\usepackage{listings}%代码
\usepackage{xcolor}%颜色
\usepackage{enumitem}%列表格式
\CTEXsetup[format+={\flushleft}]{section}

%设置文章宽度
\geometry{textwidth=18cm}
%设置页面布局
\pagestyle{plain}
\author{郑华}
\title{2018 Diary}

%正文排版开始
\begin{document} 



 \newpage

 
 
 \newpage
 \verb|Day-01 ->太古里川剧(周五、阴、22~27℃)|
 \begin{itemize}[itemindent = 2em]
 	\item 预定房间(春熙路附近)
 	\item 前往市中心 \textbf{春熙路}
 	\item 前往 \textbf{太古里} 逛一会, \textbf{负一楼小吃} , \textbf{钢管厂五区小郡肝串串香}、\textbf{翠台川菜}、\textbf{祝家川菜}(\textit{午饭})、\textbf{蜀韵蜀味火锅}(\textit{晚饭})
 	\item \textbf{方所}(全国第2家)、MUJI无印良品全球旗舰店
 	\item 芙蓉国粹\textbf{川剧}变脸 	
 	\item 回 	
 	\item 预定房间(廊桥附近)
 \end{itemize}
 
 \verb|Day-02 ->锦里川大(周六、阴、24~29℃)|
 \begin{itemize}[itemindent = 2em]
 	\item \textbf{锦里}(早上起来出发去、边吃边逛)
 	\item 杨记\textbf{乐山钵钵鸡(午饭)} 
 	\item \textbf{武伺候}(16:00 结束)
 	\item \textbf{川大}(科华北路)
 	\item \textbf{雨田饭店} 川菜(晚饭)
 	\item \textbf{廊桥}
 	\item *九眼桥酒吧
 	\item 回
 	\item 预定房间
 \end{itemize}
 
 \verb|Day-03 ->青羊区(周7、阴、25~32℃)|
 \begin{itemize}[itemindent = 2em]
 	\item \textbf{文殊院}
 	\item \textbf{青羊宫}(求签)
 	\item \textbf{杜甫草堂}		 	 	
 	\item \textbf{宽窄巷子} 吃川菜,火锅:\textit{大妙,子非,钓鱼台}
 	\item 泡桐树街、\textbf{小通巷}	、 奎星楼:\textit{二孃鸡爪爪,冒椒火辣,瓦烤}都是很不错的店
 	\item 回	 	
 \end{itemize}
 
 \verb|Day-04 ->都江堰(周1、阴转中雨、23~33℃)|
 \begin{itemize}[itemindent = 2em]
 	\item \textbf{都江堰}	 
 	\item 古镇:洛带古镇,\textbf{黄龙溪古镇},街子古镇,安仁古镇	 	
 \end{itemize}
 
 \verb|Day-05 ->熊猫、大佛(周2、小雨、23~26℃)|
 \begin{itemize}[itemindent = 2em]
 	\item *大熊猫繁育研究基地或动物园
 	\item *吃玉林串串香
 	\item \textbf{乐山大佛 }	
 	\item *青城山(前山文化,后山风景)	
 \end{itemize}
 
 \verb|Day-06 ->九寨沟(周3、阴小雨、23~27℃)|
 \begin{itemize}[itemindent = 2em]
 	\item 九寨沟		 	
 \end{itemize}
 
 \verb|tips-> |
 \begin{itemize}[itemindent = 2em]
 	\item  免费乘坐的\textbf{区域公交},其中\verb|1006路|是新晋的吃货线路,秒杀老牌\verb|154路|。每一站都有好吃的。包括\textit{皇城坝牛肉面,王婆荞面,内江冷吃兔,骨汤抄手,香满钵钵钵鸡}等等
 	\item  从\textbf{都江堰}或青城山\textbf{坐城际动车回成都}可在\textbf{犀浦站}下车,和地铁同台换乘非常方便。如果到火车北站,出站要走很久 
 	\item  \textbf{周末春熙路地铁站}人特别多,有时候会排队限流。不过你可以选择往北步行500米到市二医院站,或往东500米到东门大桥站(取决于你坐哪条线),人瞬间少很多 
 \end{itemize}
 
 \textbf{火锅榜}:\textit{骉骉火锅、巴蜀大宅门火锅、蜀九香火锅、大龙燚火锅、牛杂火锅、川西坝子、皇城老妈火锅、飘香火锅、大红锅、万州烤鱼、冒椒火辣}
 
 \textbf{小吃榜}:\textit{钵钵鸡、肥肠粉、老麻抄手、担担面、兔头、手撕烤兔、兔腰、老妈蹄花、小谭豆花、麻婆豆腐、锅盔 \& 糖油果子、蛋烘糕、凉糕、凉粉、甜水面、法式签名饼}

\end{document}