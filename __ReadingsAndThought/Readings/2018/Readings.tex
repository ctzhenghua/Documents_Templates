\documentclass[UTF8,a4paper,8pt]{ctexart} 

\usepackage{graphicx}%学习插入图
\usepackage{verbatim}%学习注释多行
\usepackage{booktabs}%表格
\usepackage{geometry}%图片
\usepackage{amsmath} 
\usepackage{amssymb}
\usepackage{listings}%代码
\usepackage{xcolor}  %颜色
\usepackage{enumitem}%列表格式
\usepackage{hyperref}
\CTEXsetup[format+={\flushleft}]{section}

\geometry{left=1.6cm,right=1.8cm,top=2cm,bottom=1.7cm} %设置文章宽度

\pagestyle{plain} 		  %设置页面布局
\author{郑华}
\title{2017读书笔记}
%代码效果定义
\definecolor{codegreen}{rgb}{0,0.6,0}
\definecolor{codegray}{rgb}{0.5,0.5,0.5}
\definecolor{codepurple}{rgb}{0.58,0,0.82}
\definecolor{backcolour}{rgb}{0.95,0.95,0.92}

\lstdefinestyle{mystyle}{
	backgroundcolor=\color{backcolour},   
	commentstyle=\color{codegreen},
	keywordstyle=\color{magenta},
	numberstyle=\tiny\color{codegray},
	stringstyle=\color{codepurple},
	basicstyle=\footnotesize,
	breakatwhitespace=false,         
	breaklines=true,                 
	captionpos=b,                    
	keepspaces=true,                 
	%numbers=left,                    
	%numbersep=5pt,                  
	showspaces=false,                
	showstringspaces=false,
	showtabs=false,                  
	tabsize=2
}
\lstset{style=mystyle, escapeinside=``}

\begin{document}          %正文排版开始
	\maketitle
	\tableofcontents
	
	\newpage
	\section{论语解读}
		\subparagraph{学而时习之}
			学而时习之,不亦说乎? 有朋自远方来,不亦说乎?人不知而不愠,不亦君子乎。
			
			首先学习什么是个问题,大千世界,万物皆可学,而孔丘所说乃为人处事。
			
			\textbf{学什么} 原来学并不是单纯的学习知识这么简单,而是惠及万千,包含了太多,年轻的时候还是不懂,当懂得时候已经不再年轻。
			
			\textbf{时习} 对于学的东西光了,那么习的范围也就不局限于复习知识了,其中就包括自己的所做所为、所经历的事情等,眼光局限了那么就可能体会不到了, 这时候就像自己翻阅笔记时,看到自己曾经做的事然后体会种种心路历程一般,受教颇深。
			
			\textbf{有朋自远方来} 人生本来得一知己少之又少,且行且珍惜
			
			\textbf{人不知而不愠} 这个知并不是知其名的知,而是理解与懂得,这是一种境界,就好比自己不被别人理解,然而却能平然处之,乃为一种气度、仁。
		
		\subparagraph{吾日三省吾身}
			曾子曰:吾日三省吾身:为人谋而不忠乎?与朋友交而不信乎?传不习乎?
			
			\textbf{为人谋而不忠乎} 替别人办事是不是尽心尽力了。
			
			\textbf{与朋友交而不信乎} 与朋友相处是不是诚实守信了。
			
			\textbf{传不习乎} 对老师传授的课程是不是用心复习了呢?
		
		\subparagraph{谨而信}
			弟子入则孝,出则弟,谨而信,泛爱众,而亲仁。行有余力,则以学文。
			
			\textbf{谨而信} 说话要谨慎,言而有信。谨慎是一种修养,只有谨慎的人,才不会对他人做出许诺,不会信口开河。一旦说出口,就要努力、尽心竭力的完成,做到言而有信。
	
	\newpage
	\section{资治通鉴故事}
		\subsection{x}
		
\end{document}