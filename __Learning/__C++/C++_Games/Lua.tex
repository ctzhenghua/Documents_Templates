\documentclass[UTF8,a4paper,8pt]{ctexart} 

\usepackage{graphicx}%学习插入图
\usepackage{verbatim}%学习注释多行
\usepackage{booktabs}%表格
\usepackage{geometry}%图片
\usepackage{amsmath} 
\usepackage{amssymb}
\usepackage{listings}%代码
\usepackage{xcolor}  %颜色
\usepackage{enumitem}%列表格式
\CTEXsetup[format+={\flushleft}]{section}


\geometry{left=1.6cm,right=1.8cm,top=2cm,bottom=1.7cm} %设置文章宽度
  
%设置页面布局
\pagestyle{plain}
\author{郑华}
\title{Lua 脚本学习}
 %代码效果定义
 \definecolor{codegreen}{rgb}{0,0.6,0}
 \definecolor{codegray}{rgb}{0.5,0.5,0.5}
 \definecolor{codepurple}{rgb}{0.58,0,0.82}
 \definecolor{backcolour}{rgb}{0.95,0.95,0.92}
 
 \lstdefinestyle{mystyle}{
 	language = {[5.1]Lua},%lua 语言指定方法
 	backgroundcolor=\color{backcolour},   
 	commentstyle=\color{codegreen},
 	keywordstyle=\color{magenta},
 	numberstyle=\tiny\color{codegray},
 	stringstyle=\color{codepurple},
 	basicstyle=\footnotesize,
 	breakatwhitespace=false,         
 	breaklines=true,                 
 	captionpos=b,                    
 	keepspaces=true,                 
 	%numbers=left,                    
 	%numbersep=5pt,                  
 	showspaces=false,                
 	showstringspaces=false,
 	showtabs=false,                  
 	tabsize=2
 }
 \lstset{style=mystyle, escapeinside=``}
 
 %正文排版开始
\begin{document} 
 	\maketitle

\newpage
\section{变量}
	\paragraph{赋值与定义}
		不需要指明类型,系统会根据给出的值进行确定
		
		除此之外可以 如下赋值
		 \begin{lstlisting}
			  a, b, c = 0, 1, 2    -->a=0, b=1, c=2
		 \end{lstlisting}
		 
	\paragraph{局部变量}
		局部变量加个 local 关键字,不是则不加
\newpage  

\section{控制结构}
	\paragraph{for}
		 \begin{lstlisting}
			 for i = 1, 10, 1 do      --初始值   结束值   步长(+x)
				 print(i)
			  end
			  
			  
		--范型 for 循环	  
			 -- print all values of array 'a'
			 for i,v in ipairs(a) do print(v) end       --i为当前的下标   v为当前下标的临时局部值
			 
			 -- print all keys of table 't
			 for k in pairs(t) do print(k) end
			 
			 --[[
			 1. 控制变量是局部变量
			 2. 不要修改控制变量的值
			 --]]
		 \end{lstlisting}

	\paragraph{while}
		 \begin{lstlisting}
			while i<x do
				print(i)
			 end
		 \end{lstlisting}
	 
	 \paragraph{if}
		  \begin{lstlisting}
			  if i<x then
				  print(i)
			   else
				   print(i+1)
				end
		  \end{lstlisting}
	\paragraph{return}
		可以返回多个变量
		 \begin{lstlisting}
			 function func(valueUsed)  
			    return  returbValueOne,returnValueTwo... -->有几个返回值写几个
		 \end{lstlisting}
		 
	\paragraph{可变参数}Lua 函数可以接受可变数目的参数,和 C 语言类似在函数参数列表中使用三点(...)
	表示函数有可变的参数。Lua 将函数的参数放在一个叫 arg 的表中,除了参数以外,arg表中还有一个域 n 表示参数的个数。
		 \begin{lstlisting}
			 --有时候我们可能需要几个固定参数加上可变参数
			 function g (a, b, ...) end
			 
			--[[ 
			 CALL PARAMETERS
			 g(3) a=3, b=nil, arg={n=0}
			 g(3, 4) a=3, b=4, arg={n=0}
			 g(3, 4, 5, 8) a=3, b=4, arg={5, 8; n=2}
			--]]
			
			--重写 print 函数:
			printResult = ""
			function print(...)
				for i,v in ipairs(arg) do
					printResult = printResult .. tostring(v) .. "\t"
				end
				printResult = printResult .. "\n"
			end
		 \end{lstlisting}
\newpage
\section{字符串}
 	 \paragraph{连接 用符号- ..   }
	 		 \begin{lstlisting}
	 		 	print("Hello" .. 'HH')    --> HelloHH
	 		 \end{lstlisting}
 		 
 	 \paragraph{字符到数字的智能转换}
	 		  \begin{lstlisting}
	 		  print("10"+11)    --> HelloHH
	 		  \end{lstlisting}
	 		  
	 \paragraph{字符串查找替换}会全部替换
			  \begin{lstlisting}
				a = "one string one"
				b = string.gsub(a,"one","other")
				
				print(a)       -->one string one
				print(b)	   -->other string other
		  	  \end{lstlisting}
	 		  
	 \paragraph{注释}
		 单行注释 是“--”
		 
		 多行注释 是“--[[      --]]”
		 
	 \paragraph{类型函数type}
		   \begin{lstlisting}
			   print(type("Hello World"))    --> string
			   print(type(10.4*3))           --> number
			   print(type(print))            --> function
			   print(type(true))             --> boolean
			   print(type(nil))              --> nil
		   \end{lstlisting}
	\paragraph{查找函数 find}
		 \begin{lstlisting}
			 s,e = string.find("Hello Lua World!","World")  -->Lua可以返回多个变量
			 			 
			 print(s,e)  --->s为目标字符串在给定字符串的起始位置,e则为终止位置   11  15
		 \end{lstlisting}
\newpage
\section{数组或表}
	\paragraph{下标从1开始不是0}
			\begin{lstlisting}
		      days={"Sunday", "Monday", "Tuesday", "Wednesday",
		      "Thursday", "Friday", "Saturday"}
		      
		      print(days[4]) --> Wednesday
		      
		      
		      tab = {sin(1), sin(2), sin(3), sin(4),
		      sin(5),sin(6), sin(7), sin(8)}
		      
		      a = {x=0, y=0} <--> a = {}; a.x=0; a.y=0
		      
		      w = {x=0, y=0, label="console"}
		      x = {sin(0), sin(1), sin(2)}
		      w[1] = "another field"
		      
		      --不管用何种方式创建 table,我们都可以向表中添加或者删除任何类型的域,构造函数仅仅影响表的初始化。
		      x.f = w							--原来没有的东西如果写出来,则自动会加进到X
		      
		      print(w["x"]) 				    --> 0  即访问的是 表中的X项,如果是w[x] -->nil
		      print(w[1]) --> another field
		      print(x.f[1]) --> another field
		      w.x = nil -- remove field "x"X
			\end{lstlisting}
	\paragraph{为什么有的东西可以用下标访问,有些不能}
			原来,只有在表不提供任何关键字时,才会按照下标进行寻找,否则,必须按照提供的关键字访问
			 \begin{lstlisting}
				 local a = {x = 12, mutou = 99, [3] = "hello"}
				 print(a["x"]);
				 
				 local a = {x = 12, mutou = 99, [3] = "hello"}
				 print(a.x);
				 
				 
				 local a = {[1] = 12, [2] = 43, [3] = 45, [4] = 90}
				 
				 --如果说,大家习惯了数组,用数字下标,又不想自己一个个数字地定义,比如:
				 local a = {12, 43, 45, 90}
				 print(a[1]);
			  \end{lstlisting}
\newpage
\section{Function}
	\paragraph{定义与使用函数}
		 \begin{lstlisting}
	 	 function fact(n)
				 if n == 0 then
					 return 1
				 else
					 return n*fact(n-1)
				 end
			end
			
			print("enter a number")
			
			a = io.read("*number")
			print(fact(a))
		 \end{lstlisting}
\newpage
\section{Lua 调用其他文件}
	\paragraph{调用Lua}
	 \begin{lstlisting}
	 --lib.lua file
	   function norm(x,y)
		    local n2 = x^2 + y^2
		    return math.sqrt(n2)
		 end
		 
		 function twice(x)
			 return 2*x
		 end
	    
	 --lua 调用file
		 dofile("lib.lua")
		 n = norm(3.4,1.0)
		 print(twice(n))
		 
	 \end{lstlisting}
\end{document} 
 		    