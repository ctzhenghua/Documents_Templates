\documentclass[UTF8,a4paper,8pt]{ctexart} 

\usepackage{graphicx}%学习插入图
\usepackage{verbatim}%学习注释多行
\usepackage{booktabs}%表格
\usepackage{geometry}%图片
\usepackage{amsmath} 
\usepackage{amssymb}
\usepackage{listings}%代码
\usepackage{xcolor}  %颜色
\usepackage{enumitem}%列表格式
\CTEXsetup[format+={\flushleft}]{section}


\geometry{textwidth=18cm} %设置文章宽度

\pagestyle{plain} 		  %设置页面布局
\author{郑华}
\title{Maya 学习}


\begin{document}          %正文排版开始
	\maketitle
	\newpage
	
\section{Maya 基本操作教程}
	\paragraph{0.添加聚关灯-观察}
	\paragraph{1.快捷键制作}
	\paragraph{2.特殊复制-实体}
	\paragraph{3.添加纹理-文件}
	\paragraph{4.旋转缩放物体}

\section{多边形 建模技术}	
	\paragraph{0.基本元素}
		\subparagraph{顶点}
		\subparagraph{边}
		\subparagraph{面}
		\subparagraph{UV点}
		\subparagraph{法线 Normal}
		
	\paragraph{1.创建菜单组}
		\subparagraph{球体等规则图形}
		\subparagraph{平面}
		\subparagraph{特殊多边形}:
		
		 圆环
		 
		 棱柱
		 
		 管道
		 
		 柏拉图多面体
	\paragraph{2.网格菜单组}
		\subparagraph{组合} 可以将选中的多个不同的多边形物体 联合成一个单独的物体,一旦合并后,只能在2个单独的模型中执行编辑
		
		 - 点击 选择菜单 -- 选择对象F8
		 
		 - shift + 要选的多个物体
		 
		 - 网格菜单 -- 组合或ctrl + G
		\subparagraph{分离}将组合的多个物体 分离为几个相互独立的多边形物体
		
		- 点击 要选择的组合对象
		
		- 网格菜单  -- 分离
		\subparagraph{提取}
		
		\subparagraph{布尔}
			组合多边形网格创建新形状,类似与数学中的交集并集差集。
			
		\subparagraph{平滑}可以通过增加网格数量来平滑选择的多边形面或物体
		
			打开“平滑选项”对话框-分段级别,越高越平滑
		\subparagraph{平均化顶点}
		\subparagraph{传递属性}
		\subparagraph{减少}
		\subparagraph{三角形化}
		\subparagraph{四角形化}
		\subparagraph{填充洞}
		\subparagraph{生成洞工具}
		\subparagraph{创建多边形工具}
		\subparagraph{雕刻几何体工具}
		\subparagraph{镜像几何体}
		
	\paragraph{3.编辑网格菜单组}
		\subparagraph{保持面的连接性添加边的分段数合并边工具}
		\subparagraph{挤出滑动边工具删除边/顶点}
		\subparagraph{桥接变换组件切角顶点}
		\subparagraph{附加到多边形工具翻转三角形边倒角}
		\subparagraph{在网格上投影曲线刺破面}
		\subparagraph{使用投影的曲线分割网格的栔形面}
		\subparagraph{切割面工具复制面}
		\subparagraph{交互式分割工具连接组件}
		\subparagraph{插入循环边工具合并}
		\subparagraph{偏移循环边工具收拢}
		
	\paragraph{4.代理菜单组}
		\subparagraph{细分曲面代理}
		\subparagraph{移除细分曲面代理镜像}
		\subparagraph{折痕工具}
		\subparagraph{切换代理显示}
		\subparagraph{代理和细分曲面同时显示}
	
	\paragraph{5.法线菜单组}
		\subparagraph{反向}
		\subparagraph{软化边/硬化边}
		\subparagraph{设置法线角度}
\section{NURBS 建模技术}
	\paragraph{1.Nurbs 曲线}
		\subparagraph{Nurbs 曲线的基本概念}
		
		\subparagraph{Nurbs 曲线的构成要素}
		
		\subparagraph{Nurbs 曲线的创建}
		
		\subparagraph{Nurbs 曲线的精度}
	\paragraph{2.Nurbs 曲面}
		\subparagraph{Nurbs 曲面的构成}
		
		\subparagraph{Nurbs 曲面的创建方法}
		
	\paragraph{+Nurbs 曲面曲线的参数化}
	
	\paragraph{3.创建菜单组}
		\subparagraph{球体等规则体}
	\paragraph{4.编辑曲线菜单组}
		\subparagraph{复制曲面曲线}
		\subparagraph{附加曲线}
		\subparagraph{分离曲线}
		\subparagraph{对齐曲线}	
		\subparagraph{开发/闭合曲线}
		\subparagraph{移动接缝}
		\subparagraph{切割曲线}
		\subparagraph{曲线相交}
		\subparagraph{曲线圆角}
		\subparagraph{插入结}
		\subparagraph{延伸曲线}
		\subparagraph{重建曲线}
		\subparagraph{添加点工具}
		
	\paragraph{5.曲面菜单组}	
		\subparagraph{旋转}
		\subparagraph{放样}
		\subparagraph{平面}
		\subparagraph{挤出}
		\subparagraph{双轨成型}
		\subparagraph{边界}
		\subparagraph{方形}
		\subparagraph{倒角}
		\subparagraph{倒角+}
		
	\paragraph{6.编辑Nurbs 菜单组}
		\subparagraph{复制Nurbs 面片}
		\subparagraph{在曲线上 投影曲线}
		\subparagraph{曲面相交}
		\subparagraph{修剪工具/取消修剪}
		\subparagraph{布尔}
		\subparagraph{附件曲面}
		\subparagraph{附件而不移动}
		\subparagraph{分离曲面}
		\subparagraph{对齐曲面}
		\subparagraph{开放/闭合 曲面}
		\subparagraph{移动接缝}
		\subparagraph{插入等参线}
		\subparagraph{延伸曲面}
		\subparagraph{偏移曲面}
		\subparagraph{反转曲面方向}
		\subparagraph{重建曲面}
		\subparagraph{圆化工具}

\section{建模案例}
	\subsection{头部建模}
	http://www.fevte.com/tutorial-9952-1.html
		\paragraph{1.建立对应的参考坐标系}

\end{document} 
