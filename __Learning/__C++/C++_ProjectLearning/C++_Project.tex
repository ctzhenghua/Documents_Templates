\documentclass[UTF8,a4paper,8pt]{ctexbook} 

 \usepackage{graphicx}%学习插入图
 %\graphicspath{{figures/}}
 \usepackage{verbatim}%学习注释多行
 \usepackage{booktabs}%表格
 \usepackage{geometry}%图片
 \usepackage{amsmath} 
 \usepackage{amssymb}
 \usepackage{listings}%代码
 \usepackage{xcolor}  %颜色
 \usepackage{enumitem}%列表格式
 \usepackage{hyperref}
 \CTEXsetup[format+={\flushleft}]{section}

\graphicspath{{figure/}}
\geometry{left=1.6cm,right=1.8cm,top=2cm,bottom=1.7cm} %设置文章宽度

\pagestyle{plain} 		  %设置页面布局
\author{郑华}
\title{C++  服务器案例学习笔记}
 %代码效果定义
 \definecolor{mygreen}{rgb}{0,0.6,0}
 \definecolor{mygray}{rgb}{0.5,0.5,0.5}
 \definecolor{mymauve}{rgb}{0.58,0,0.82}
 \lstset{ %
 	backgroundcolor=\color{white},   % choose the background color
 	basicstyle=\footnotesize\ttfamily,        % size of fonts used for the code
 	%stringstyle=\color{codepurple},
 	%basicstyle=\footnotesize,
 	%breakatwhitespace=false,         
 	%breaklines=true,                 
 	%captionpos=b,                    
 	%keepspaces=true,                 
 	%numbers=left,                    
 	%numbersep=5pt,                  
 	%showspaces=false,                
 	%showstringspaces=false,
 	%showtabs=false,        
 	columns=fullflexible,
 	breaklines=true,                 % automatic line breaking only at whitespace
 	captionpos=b,                    % sets the caption-position to bottom
 	tabsize=4,
 	commentstyle=\color{mygreen},    % comment style
 	escapeinside={\%*}{*)},          % if you want to add LaTeX within your code
 	keywordstyle=\color{blue},       % keyword style
 	stringstyle=\color{mymauve}\ttfamily,     % string literal style
  	frame=single,					%tb top and bottom; L left double line
  	xleftmargin=.06\textwidth, 
  	%xrightmargin=.1\textwidth,
 	rulesepcolor=\color{red!20!green!20!blue!20},
 	% identifierstyle=\color{red},
 	language=c++,
 }

\begin{document}          %正文排版开始
 	\maketitle
 	\tableofcontents
\chapter{开源-muduo-1.1.2}	
	 \section{架构图}
		 \begin{figure}[h]
		 	\centering
		 	\includegraphics[scale = 0.7]{muduo.png}
		 	\caption{架构图}
		 \end{figure}
 
	 \section{参考}
		 代码架构:见muduo 参考手册,net 文件夹下
		 
		 博客:\url{http://www.cppblog.com/Solstice/default.html?page=3}
		 
			 \url{https://www.oschina.net/question/28_61182}
			 
	\section{基础类设计}
		\subsection{Thread}
			\begin{figure}[ht]
				\centering
				\includegraphics[scale= 0.3]{figure/muduoThread.png}
				\caption{Thread类图}
			\end{figure}
			
			\subsubsection{Thread头文件}
				\begin{lstlisting}
	namespace muduo
	{
		
		class Thread : boost::noncopyable
		{
		  public:
			typedef boost::function<void ()> ThreadFunc;
			
			explicit Thread(const ThreadFunc&, const string& name = string());
			#ifdef __GXX_EXPERIMENTAL_CXX0X__
			explicit Thread(ThreadFunc&&, const string& name = string());
			#endif
			~Thread();
			
			void start();
			int join(); // return pthread_join()
			
			bool started() const { return started_; }

			pid_t tid() const { return *tid_; }
			const string& name() const { return name_; }
			
			static int numCreated() { return numCreated_.get(); }
			
		  private:
			void setDefaultName();
			
			bool       started_;
			bool       joined_;
			pthread_t  pthreadId_;
			boost::shared_ptr<pid_t> tid_;
			ThreadFunc func_;
			string     name_;
			
			static AtomicInt32 numCreated_;
		};
		
	}
				\end{lstlisting}
			
			\subparagraph{要点}	
				\begin{itemize}[itemindent = 2em]
					\item 该线程类使用基于对象的方式实现,即 注册回调函数 来实现各对象不同的功能。
					\item 静态方法不能调用非静态,如this 
					\item 参数类型匹配,例如传递对象指针先得转化为void*, 进去后然后再转回来。
					\item \verb|pthread_create|在线程创建以后,就开始运行相关的线程函数
					\item \verb|pthread_join|如果没有加\verb|pthread_join()|方法,main线程里面直接就执行起走了,加了之后是等待线程执行了之后才执行的后面的代码。
					
					\url{http://blog.csdn.net/dinghqalex/article/details/42921931}
				\end{itemize}
				
				回调图如下\ref{callback}:
				\begin{figure}[htbp]
					\centering
					\includegraphics[scale= 0.7]{figure/muduoCallback.png}
					\label{callback}
					\caption{回调演示}
				\end{figure}
				
		\subsection{MutexLock}
			\begin{figure}[htbp]
				\centering
				\includegraphics[scale= 0.4]{figure/muduoMutexlock.png}
				\caption{MutexLock类图}
			\end{figure}
			
			\subparagraph{要点}
				\begin{itemize}[itemindent = 2em]
					\item 存在锁竞争
					\item 
				\end{itemize}
				
		\subsection{Condition}
			
			\begin{figure}[htbp]
				\centering
				\includegraphics[scale= 0.4]{figure/muduoCondition.png}
				\caption{Condition类图}
			\end{figure}
				
			
			\subparagraph{具体流程}
				\begin{itemize}[itemindent = 2em]
					\item 锁住 \verb|mutex_lock|
					\item 等待条件满足
						\begin{lstlisting}[frame=L]
	while()
	{
		mutex_unlock
		等待条件
		mutex_lock
	}
						\end{lstlisting}
					\item 解锁 \verb|mutex_unlock|
				\end{itemize}
				
			\subparagraph{要点}
				\begin{itemize}
					\item 观察者模式
				\end{itemize}
				
		
		\subsection{CountDownLatch}
				可以用于所有子线程等待主线程发起“起跑”
				
				可以用于主线程等待子线程初始化完毕才开始工作。
				
				\begin{figure}[htbp]
					\centering
					\includegraphics[scale= 0.4]{figure/muduoCountdownlatch.png}
					\caption{Condition类图}
				\end{figure}
		
		\subsection{BlockingQueue 与 BoundedBlockingQueue}
			
				\begin{figure}[htbp]
					\centering
					\includegraphics[scale= 0.7]{figure/BoundedBlockingQueue.png}
					\caption{有界队列}
				\end{figure}
				
			\subsubsection{BlockingQueue<T>}
				\begin{figure}[htbp]
					\centering
					\includegraphics[scale= 0.6]{figure/BlockingQueue.png}
					\caption{BlockingQueue类图}
				\end{figure}
				
			\subsubsection{BoundedBlockingQueue<T>}
				\begin{figure}[htbp]
					\centering
					\includegraphics[scale= 0.4]{figure/BoundBlockingQueue.png}
					\caption{BlockingQueue}
				\end{figure}
				
				
				\subparagraph{实现示例}\verb|->|
					\begin{lstlisting}
	namespace muduo
	{
		
		template<typename T>
		class BlockingQueue : boost::noncopyable
		{
			public:
			BlockingQueue()
			: mutex_(),
			notEmpty_(mutex_),
			queue_()
			{
			}
			
			void put(const T& x)
			{
				MutexLockGuard lock(mutex_);
				queue_.push_back(x);
				notEmpty_.notify(); // wait morphing saves us
				// http://www.domaigne.com/blog/computing/condvars-signal-with-mutex-locked-or-not/
			}
			
			#ifdef __GXX_EXPERIMENTAL_CXX0X__
			void put(T&& x)
			{
				MutexLockGuard lock(mutex_);
				queue_.push_back(std::move(x));
				notEmpty_.notify();
			}
			// FIXME: emplace()
			#endif
			
			T take()
			{
				MutexLockGuard lock(mutex_);
				// always use a while-loop, due to spurious wakeup
				while (queue_.empty())
				{
					notEmpty_.wait();
				}
				assert(!queue_.empty());
				#ifdef __GXX_EXPERIMENTAL_CXX0X__
				T front(std::move(queue_.front()));
				#else
				T front(queue_.front());
				#endif
				queue_.pop_front();
				return front;
			}
			
			size_t size() const
			{
				MutexLockGuard lock(mutex_);
				return queue_.size();
			}
			
			private:
			mutable MutexLock mutex_;
			Condition         notEmpty_;
			std::deque<T>     queue_;
		};
		
	}
					\end{lstlisting}
				
			\subsubsection{核心要点}
				\begin{itemize}
					\item 对于 无界队列只需要将 notFull 条件变量去掉即可
				\end{itemize}
		\subsection{ThreadPool}
				\begin{figure}[htbp]
					\centering
					\includegraphics[scale= 0.7]{figure/ThreadPool.png}
					\includegraphics[scale= 0.5]{figure/muduoThreadPool.png}
					\caption{ThreadPool模型}
				\end{figure}
			
			\subparagraph{示例代码}\url{https://github.com/ctzhenghua/muduo/blob/master/muduo/base/ThreadPool.cc}
			
			\url{http://blog.csdn.net/u013507368/article/details/48130151}
		
		\subsection{Singleton}	
				\begin{figure}[htbp]
					\centering
					\includegraphics[scale= 0.4]{figure/muduoSingleton.png}
					\caption{ThreadPool模型}
				\end{figure}
			\subsubsection{核心要点}
				\begin{itemize}
					\item \verb|pthread_once|  确定某个函数只调用一次
					\item \verb|atexit|  确定对象销毁时执行的函数
					\item \verb|typedef char T_must_be_complete_type[sizeof(T) == 0?-1,1]|
				\end{itemize}
			\subparagraph{示例代码}\url{https://github.com/ctzhenghua/muduo/blob/master/muduo/base/Singleton.h}
		
		\subsection{线程特定数据}
			\begin{itemize}
				\item 在单线程程序中,我们经常要用到"全局变量"以实现多个函数间共享数据
				\item 在多线程环境下,由于数据空间是共享的,因此全局变量也为所有线程所共有。 
				\item 但有时应用程序设计中有必要提供线程私有的全局变量,仅在某个线程中有效,但却可以跨多个函数访问。
				\item POSIX线程库通过维护一定的数据结构来解决这个问题.(如共享内存一般,创建key,然后使用,如下图)
				\item 线程特定数据也称为线程本地存储Thread-local storage.对于POD类型的线程本地存储,可以用\verb|__thread|关键字
			\end{itemize}	
			
				\begin{figure}[htbp]
					\centering
					\includegraphics[scale= 0.5]{figure/threadLocal.png}
					\caption{ThreadLocal 用法及模型}
				\end{figure}
				
		\subsection{日志}
			\subparagraph{作用}
				\begin{itemize}
					\item 开发时:
						\begin{itemize}
							\item 调试错误
							\item 更加的理解程序
						\end{itemize}
					
					\item 运行时:
						\begin{itemize}
							\item 诊断系统故障并处理
							\item 记录系统运行状态
						\end{itemize}
				\end{itemize}
			\subparagraph{日志级别}
				\begin{itemize}
					\item \verb|TRACE|
					指出比\verb|DEBUG|粒度更细的一些信息事件(开发过程中使用)
					
					\item \verb|DEBUG|
					指出细粒度信息事件对调试应用程序是非常有帮助的。(开发过程中使用)
					
					\item \verb|INFO|
					表明消息在粗粒度级别上突出强调应用程序的运行过程
					
					\item \verb|WARN|
					系统能正常运行,但可能会出现潜在错误的情形。
					
					\item \verb|ERROR|
					指出虽然发生错误事件,但仍然不影响系统的继续运行。
					
					\item \verb|FATAL|
					指出每个严重的错误事件将会导致应用程序的退出
				\end{itemize}
	\section{网络类设计}
		
\chapter{简单http服务器}
	\section{架构图}
		\begin{figure}[h]
			\centering
			\includegraphics[scale = 0.5]{figure/httpServer.png}
			\caption{架构图}
		\end{figure}
	
	\section{http 协议}
			客户端请求的文件格式,\url{https://www.cnblogs.com/lexiaofei/p/6943690.html}
		\subsection{请求服务器数据}
			\begin{itemize}
				\item \verb|HTTP请求方法| 
					\begin{itemize}
						\item   \verb|Http|协议定义了很多与服务器交互的方法,最基本的有4种,分别是\verb|GET|,\verb|POST|,\verb|PUT|,\verb|DELETE|.
						
						\item 	一个\verb|URL|地址用于描述一个网络上的资源,而HTTP中的\verb|GET|,\verb| POST|, \verb|PUT|, \verb|DELETE|就对应着对这个资源的查,改,增,删4个操作。我们最常见的就是\verb|GET|和\verb|POST|了。GET一般用于获取/查询资源信息,而\verb|POST|一般用于更新资源信息.
						
						\item  协议版本
					\end{itemize}
				
				\verb|GET /3.txt  HTTP/1.1 |
				\item 服务器地址
				
					\verb|Host: 192.168.0.3:80|
				\item 协议头部分(可选)
				\item 协议头结束(空行)
					
					\verb|\r\n|
			\end{itemize}
		\subsection{服务器应答浏览器}
			\begin{itemize}
				\item \verb|HTTP应答方法| 
				
				\verb|HTTP/1.1  状态码 OK ->  HTTP/1.1 200 OK|
				\item 内容格式(必选项)
				
				\verb|Content-Type: text/plain; charset=iso-8859-1|
					\begin{itemize}
						\item \verb|*.html :text/html; charset=iso-8859-1| 
						\item \verb|*.jpg : image/jpeg|
						\item \verb|*.png : image/png|
					\end{itemize}
				\item 协议结束空行
				
				\verb|\r\n|
				\item 内容
			\end{itemize}
	\section{HTML}
		\begin{lstlisting}
	<html>
		<head><title> TestPage </title> </head>
		<body>
			<p> Test Ok </p>
			<img src='xx.jpg' /img>
		</body>
	</html>
		\end{lstlisting}	
	
	\section{xinetd}
		\subsection{守护进程}
			daemon,系统中的一个后台进程,周期性的执行某个任务,或者等待某个事件的发生。不会随用户的注销而退出。
			
			\subparagraph{创建守护进程} 不能是组长进程(父进程)
				\begin{enumerate}[itemindent = 2em]
					\item fork子进程,父进程结束
					\item 子进程setsid() 创建新会话,脱离终端控制
					\item 修订权限、文件描述符 重定向dump2
				\end{enumerate}
		
		\subsection{配置xinetd}
			\subparagraph{举例}\verb|xinetd| 接受到客户端请求后,启动 zhangsan 的可执行文件  \verb|./zhangsan|
				\begin{enumerate}[itemindent = 2em ]
					\item 在\verb|/etc/xinetd.d| 创建一个名为 zhangsan 的文件(配置文件)
					\item 写入配置内容, \verb|service zhangsan{}|
					\item \verb|server = /home/itcast/zhangsan |(zhangsan 应具有可执行属性,并在该目录下)
					\item \verb|vi /etc/services |
					
						加入\verb|zhangsan 2222/tcp #myhttpServer|
						
					\item 重启xinetd 服务器\verb|sudo service xinetd restart stop start restart|
					\item \verb|ps aux | grep xinetd|
					
					\item 浏览器地址中输入\verb|localhost:2222|
				\end{enumerate}
				
\chapter{开源-libevent-1.4}
	\section{参考}
		代码架构:见参考手册,net 文件夹下
	
\chapter{开源-lighthttpd-1.4.15}
	\section{架构图}
		\begin{figure}[h]
			\centering
			\includegraphics[scale = 0.5]{lighttpd.jpg}
			\caption{架构图}
		\end{figure}
\chapter{开源-nginx-0.5.38}
	\section{架构}
		\begin{figure}[h]
			\centering
			\includegraphics[scale = 0.7]{nginx.png}
			\caption{架构图}
		\end{figure}
	\section{参考}
		代码架构:\url{http://tengine.taobao.org/book/chapter_09.html}

\chapter{游戏服务器 - 戴汉水}    
	\section{经验学习}
		 \subsection{\#ifdef .. 编译选择}
		 
		 \subsection{string 和 wstring 相互转换}
		 
		 \subsection{GBK 和 UTF8 的相互变换}
		 
		 \subsection{boost 智能指针}
		 
		 \subsection{游戏服务器种类}
		 数据库服务器, 游戏服务器, 登陆服务器, battle 服务器
		 
		 \subsection{enum 枚举的使用}
		 
		 \subsection{按模块编码,而不是头文件与cpp 文件}有点像java包的概念,像包一样的分解程序,分解成lib,然后调用,继承
		 
\newpage
	\section{错误记录}
		\subsection{错误	204	error LNK2001: 无法解析的外部符号} 
		\verb|"__declspec(dllimport)| 
		
		\verb|class std::basic_ostream<char,struct std::char_traits<char> >| 
		
		\verb|& __cdecl std::endl(class std::basic_ostream<char,struct std::char_traits<char> > &)"| 
		
		\subparagraph{解决办法}
		VS20xx:
		
		项目、属性、链接器、常规、附加库目录:填写附加依赖库所在目录 分号间隔多项
		
		项目、属性、链接器、输入、附加依赖项:填写附加依赖库的名字.lib 空格或分号间隔多项



\chapter{游戏服务器2 - 戴汉水}
	\section{经验}
	
		\subsection{设置自己的 解决方案目录结构}:\url{http://www.360doc.com/content/13/0422/10/8251840_280065154.shtml}
	
	
		\subsection{学会程序 跟踪,与Debug}
	\section{错误记录}
		\subsection{Error MSB3073 代码9009}:
		
		\subsection{Error fatal error LNK1104: cannot open file 'atlsd.lib'}:\url{http://blog.csdn.net/hsluoyc/article/details/46312293}
		
		貌似是 2008以前版本用的库有个 atlsd.lib, 由于atlsd.lib在高版本VS中也不存在,因此只好用其Release版本代替,经测试可以使用。没有安装高版本VS的同学可以留言索取atls.lib。
		
		
		\subsection{Project : error PRJ0019: 工具从"正在执行预生成事件..."}
			配置属性-生成事件-预编译事件或生成后事件-命令行,删除命令行内容即可
			
			
		\subsection{subwcrev 不是内部或外部命令,也不是可运行的程序}:
			svn 命令: 该版本visual studio  未安装SVN 版本控制器, 导致的错误
			
		\subsection{resource 文件目录如果打不开}请检查是不是不存在.rc文件,或.rc文件多了后缀如.rc.in
				
\chapter{游戏解决方案的搭建}
	\section{DLL动态库}
	
	\section{调用DLL}
	
	\section{调用DLL中的头文件,但其实不是}	


\chapter{Others}
	\section{线程池}多线程技术主要解决处理器单元内多个线程执行的问题,它可以显著减少处理器单元的闲置时间,增加处理器单元的吞吐能力。
	
		线程池的好处就在于线程复用,一个任务处理完成后,当前线程可以直接处理下一个任务,而不是销毁后再创建,非常适用于连续产生大量并发任务的场合。
			
		\url{http://wenku.baidu.com/view/f0bd6127ccbff121dd36831a.html}
	
		\subsection{核心思想} 假设一个服务器完成一项任务所需时间为:T1 创建线程时间,T2 在线程中执行任务的时间,T3 销毁线程时间。
		
		如果:T1 + T3 远大于 T2,则可以采用线程池,以提高服务器性能。
		
		\textbf{线程池技术正是关注如何缩短或调整T1,T3时间的技术,从而提高服务器程序性能的},它把T1,T3分别安排在服务器程序的启动和结束的时间段或者一些空闲的时间段,这样在服务器程序处理客户请求时,不会有T1,T3的开销了。
		
		除此之外,线程池能够减少创建的线程个数。通常线程池所允许的并发线程是有上界的,如果同时需要并发的线程数超过上界,那么一部分线程将会等待。而传统方案中,如果同时请求数目为2000,那么最坏情况下,系统可能需要产生2000个线程。尽管这不是一个很大的数目,但是也有部分机器可能达不到这种要求。  
		
		因此线程池的出现正是着眼于减少线程池本身带来的开销。线程池采用预创建的技术,在应用程序启动之后,将立即创建一定数量的线程(N1),放入空闲队列中。这些线程都是处于阻塞Suspended)状态,不消耗CPU,但占用较小的内存空间。当任务到来后,缓冲池选择一个空闲线程,把任务传入此线程中运行。当N1个线程都在处理任务后,缓冲池自动创建一定数量的新线程,用于处理更多的任务。在任务执行完毕后线程也不退出,而是继续保持在池中等待下一次的任务。当系统比较空闲时,大部分线程都一直处于暂停状态,线程池自动销毁一部分线程,回收系统资源。  
		
		基于这种预创建技术,线程池将线程创建和销毁本身所带来的开销分摊到了各个具体的任务上,执行次数越多,每个任务所分担到的线程本身开销则越小,不过我们另外可能需要考虑进去线程之间同步所带来的开销。
		\subsection{组成部分}
			\subparagraph{1.线程池管理器 ThreadPool}用于创建并管理线程池,包括 创建线程池,销毁线程池,添加新任务
			
			\subparagraph{2.工作线程 PoolWorker}线程池中线程,在没有任务时处于等待状态,可以循环的执行任务
			
			\subparagraph{3.任务接口 Task}每个任务必须实现的接口,以供工作线程调度任务的执行,它主要规定了任务的入口,任务执行完后的收尾工作,任务的执行状态等
			
			\subparagraph{4.任务队列 taskQueue}用于存放没有处理的任务。提供一种缓冲机制。
		
		\subsection{实现}
			\subparagraph{C++}\url{http://wenku.baidu.com/view/a4cc1093daef5ef7ba0d3c3a.html}
			
			\subparagraph{Boost}\url{http://www.oschina.net/code/snippet_170948_18231}
		
		
\end{document} 
 		    