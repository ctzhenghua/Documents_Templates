\documentclass[UTF8,a4paper,12pt]{ctexbook} 

\usepackage{graphicx}%学习插入图
\usepackage{verbatim}%学习注释多行
\usepackage{booktabs}%表格
\usepackage{geometry}%图片
\usepackage{amsmath}
\usepackage{amssymb}
\usepackage{listings}%代码
\usepackage{xcolor}  %颜色
\usepackage{enumitem}%列表格式
\setenumerate[1]{itemsep=0pt,partopsep=0pt,parsep=\parskip,topsep=5pt}
\setitemize[1]{itemsep=0pt,partopsep=0pt,parsep=\parskip,topsep=5pt}
\setdescription{itemsep=0pt,partopsep=0pt,parsep=\parskip,topsep=5pt}
\usepackage{tcolorbox}
\usepackage{algorithm}  %format of the algorithm
\usepackage{algorithmic}%format of the algorithm
\usepackage{multirow}   %multirow for format of table
\usepackage{tabularx} 	%表格排版格式控制
\usepackage{array}	%表格排版格式控制
\usepackage{hyperref} %超链接 \url{URL}
\usepackage{tikz}
\usepackage{dirtree}


\usetikzlibrary{intersections,
	positioning,
	petri,
	backgrounds,
	fit,
	decorations.pathmorphing,
	arrows,
	arrows.meta,
	bending,
	calc,
	intersections,
	through,
	backgrounds,
	shapes.geometric,
	quotes,
	matrix,
	trees,
	shapes.symbols,
	graphs,
	math,
	patterns,
	external}
\CTEXsetup[format+={\flushleft}]{section}

%%%% 设置图片目录
\graphicspath{{figure/}}

%%%% 段落首行缩进两个字 %%%%
\makeatletter
\let\@afterindentfalse\@afterindenttrue
\@afterindenttrue
\makeatother
\setlength{\parindent}{2em}  %中文缩进两个汉字位

%%%% 下面的命令重定义页面边距,使其符合中文刊物习惯 %%%%
\addtolength{\topmargin}{-54pt}
\setlength{\oddsidemargin}{0.63cm}  % 3.17cm - 1 inch
\setlength{\evensidemargin}{\oddsidemargin}
\setlength{\textwidth}{14.66cm}
\setlength{\textheight}{24.00cm}    % 24.62

%%%% 下面的命令设置行间距与段落间距 %%%%
\linespread{1.4}
\setlength{\parskip}{0.5\baselineskip}
\geometry{left=1.6cm,right=1.8cm,top=2cm,bottom=1.7cm} %设置文章宽度
\pagestyle{plain} 		  %设置页面布局

%代码效果定义
\definecolor{mygreen}{rgb}{0,0.6,0}
\definecolor{mygray}{rgb}{0.5,0.5,0.5}
\definecolor{mymauve}{rgb}{0.58,0,0.82}
\lstset{ %
	backgroundcolor=\color{white},   % choose the background color
	basicstyle=\footnotesize\ttfamily,      % size of fonts used for the code
	%stringstyle=\color{codepurple},
	%basicstyle=\footnotesize,
	%breakatwhitespace=false,         
	%breaklines=true,                 
	%captionpos=b,                    
	%keepspaces=true,                 
	%numbers=left,                    
	%numbersep=5pt,                  
	%showspaces=false,                
	%showstringspaces=false,
	%showtabs=false,        
	columns=fullflexible,
	breaklines=true,                 % automatic line breaking only at whitespace
	captionpos=b,                    % sets the caption-position to bottom
	tabsize=4,
	commentstyle=\color{mygreen},    % comment style
	escapeinside={\%*}{*)},          % if you want to add LaTeX within your code
	keywordstyle=\color{blue},       % keyword style
	stringstyle=\color{mymauve}\ttfamily,     % string literal style
	frame=L,
	xleftmargin = .079\textwidth,
	rulesepcolor=\color{red!20!green!20!blue!20},
	% identifierstyle=\color{red},
	language=c++,
}
 \author{\kaishu 郑华}
 \title{\heiti U3D笔记}
 
\begin{document}          %正文排版开始
 	\maketitle
 	\tableofcontents

\chapter{基础}
	入门参考:\url{https://unity3d.com/cn/learn/tutorials}
	
	\section{如何将脚本与具体对象绑定}
		\begin{enumerate}
			\item 右键\verb|asset|文件夹,创建C\#脚本
			\item 编写脚本
			\item 将\verb|asset |中的脚本拖拽到 \verb|hiearch |视图中的\verb|MainCamera |中
			\item 如果脚本是作用于场景中的某个物体,\textbf{则将该脚本拖拽到该物体上}
		\end{enumerate}
	
	\section{序列化-  [SerializedField]}
		\textit{通常情况下},GameObject上挂的MonoBehaviour脚本中的\textbf{私有变量}\textit{不会显示在Inspector面板上},即不会被序列化。
		
		\textbf{但如果指定了SerializedFiled特性},就可以被序列化了。
		
		\begin{lstlisting}
	public class Test : MonoBehaviour 
	{
		public string Name;
		[SerializeField]
		private int Hp; 
	}
		\end{lstlisting}
		
		\begin{figure}[H]
			\centering
			\includegraphics[scale=0.8]{SerilizedFiled.jpg}
			\caption{序列化操作 -在Inspector上显示}
		\end{figure}
	
	\section{常用技巧}
		\begin{itemize}
			\item \verb|ctrl + d| 复制
			\item \verb|shift + 鼠标 | 等比例缩放 
			\item \verb|shift + alt + 鼠标 | 原地等比例缩放
			\item \verb|在Unity 编辑器中输入汉字 | 需要借助其他文本拷贝粘贴
			\item \verb|q、w、e、r、t |在操作UI时尽量使用 T,以避免z轴发生的变化 
		\end{itemize}
	
	\section{MonoBehaviour 生命周期、渲染管线}
		
		\subsection{脚本渲染流程}
			\begin{figure}[H]
				\centering
				\includegraphics[scale=0.44]{scriptLifeCircle.jpg}
				\caption{脚本生命周期核心方法}
			\end{figure}
			
			\clearpage
			\begin{figure}[H]
				\centering
				\includegraphics[scale=0.78]{LifeCicle.png}
				\caption{简要核心方法}
			\end{figure}
		
		update:当其\textbf{所在的物体}属于未激活的话(\verb|active为false|),\textit{该物体上所有脚本中包含的协程代码都是不会被执行的}。
		\subsection{核心方法}
			\begin{enumerate}
				\item \verb|Reset |:
				\item \verb|Awake |:
				\item \verb|OnEnable |:
				\item \verb|Start |:
				\item \verb|FixedUpdate |:
				\item \verb|yield WaitForFixedUpdate |:
				\item \verb|OnTriggerXXX |:
				\item \verb|Update |:
				\item \verb|LateUpdate |:
				\item \verb|OnWillRenderObject |:
				\item \verb|OnGUI |:
				\item \verb|yield WaitForEndOfFrame |:
				\item \verb|OnDisable |:
				\item \verb|OnDestroy |:
			\end{enumerate}
		
	\section{Unity 委托}
		
		\paragraph{定义}\verb|public delegate void MyDelegate(int num);|
		
			\textbf{委托}就是\verb|C#|封装的\textbf{C++的函数指针}。
			
			定义一个委托MyDelegate,如同定义一个类一样,此时的委托没有经过实例化是无法使用的,而他的实例化必须接收一个返回值和参数都与他等同的函数,此处的委托MyDelegate只能接收返回值为void,参数为一个int的函数
			
		\paragraph{实例化委托}:\verb|MyDelegate _MyDelegate=new MyDelegate(TestMod);|
			
			以\verb|TestMod|函数实例化一个\verb|MyDelegate|类型的委托\verb|_MyDelegate|,此处\verb|TestMod|函数的定义就应如下:
				
			\verb|public void TestMod(int _num);|
				
			之后调用\verb|_MyDelegate(100)|时就完全等同于调用\verb|TestMod(100)|
		
		
		
	\section{Unity 协程}
		\subsection{开启方式}
			协程:协同程序,在主程序运行的同时,开启另外一段逻辑处理,来协同当前程序的执行。
			
			\paragraph{StartCoroutine(string MethodName)}
				\begin{itemize}
					\item 参数是方法名					
					\item 形参方法可以有返回值
				\end{itemize}
			
			\paragraph{StartCoroutine(IEnumerator method)}
				\begin{itemize}
					\item 参数是方法名(\verb|TestMethod()|),方法中\textbf{可以包含多个参数}
					\item \verb|IEnumrator| 类型的方法不能含有\verb|ref或者out| 类型的参数,\textbf{但可以含有被传递的引用}
					\item \textbf{必须有有返回值},且返回值类型为\verb|IEnumrator|,返回值使用(\textit{yield retuen +表达式或者值},或者 \textit{yield break})语句	
			\end{itemize}
		
		\subsection{终止方式}
			\paragraph{StopCoroutine(string MethodName)}
				只能终止指定的协程
			
			\paragraph{StopAllCoroutine()}
				终止所有协程
				
		\subsection{yield 方式}
			\paragraph{yield return}
				挂起,\textbf{程序遇到}\verb|yield|\textbf{关键字时会被挂起},暂停执行,\textbf{等待条件满足时从当前位置继续执行}
				
				\begin{itemize}
					\item \verb|yield return 0| or \verb|yield return null|:程序在下一帧中从当前位置继续执行
					\item \verb|yield return 1,2,3,......|: 程序等待1,2,3...帧之后从当前位置继续执行
					\item \verb|yield return new WaitForSeconds(n)|:程序等待n秒后从当前位置继续执行
					\item \verb|yield new WaitForEndOfFrame()|:在所有的渲染以及GUI程序执行完成后从当前位置继续执行
					\item \verb|yield new WaitForFixedUpdate()|:所有脚本中的FixedUpdate()函数都被执行后从当前位置继续执行
					\item \verb|yield return WWW()|:等待一个网络请求完成后从当前位置继续执行
					\item \verb|yield return StartCoroutine()|:等待一个协程执行完成后从当前位置继续执行
				\end{itemize}
			
			\paragraph{yield break}
				如果使用\verb|yield break|语句,将会导致\textbf{如果协程的执行条件不被满足},\textit{不会从当前的位置继续执行程序,而是直接从当前位置跳出函数体,回到函数的根部}
				
				\color{blue}相当于:\textbf{return; + 暂停}\color{black}
				
		\subsection{执行原理}
			协程函数的返回值是\verb|IEnumerator|,它是一个迭代器,\textbf{可以把它当成执行一个序列的某个节点的指针},它提供了两个重要的接口,分别是\verb|Current|(返回当前指向的元素)和\verb|MoveNext()|(将指针向后移动一个单位,\textit{如果移动成功,则返回true})
			
			\verb|yield|关键词\textbf{用来声明序列中的下一个值或者是一个无意义的值},如果使用\verb|yield return x|(x是指一个具体的对象或者数值)的话,那么\verb|MoveNext|返回为\verb|true|并且\verb|Current|被赋值为\verb|x|,如果使用\verb|yield break|使得\verb|MoveNext()|返回为\verb|false|
			
			如果\verb|MoveNext|函数返回为\verb|true|\textbf{意味着协程的执行条件被满足,则能够从当前的位置继续往下执行}。否则不能从当前位置继续往下执行。	

	\paragraph{委托+协程}
		\url{https://blog.csdn.net/qq992817263/article/details/51514449}
		
		\begin{itemize}
			\item 实现延时
			\item 实现给定函数传参
			\item 实现特定功能
		\end{itemize}
	
		\begin{lstlisting}
	// 延时执行

	// <param name="action">执行的委托</param>
	// <param name="obj">委托的参数</param>
	// <param name="delaySeconds">延时等待的秒数</param>
	public IEnumerator DelayToInvokeDo(Action<GameObject> action, GameObject obj,float delaySeconds)
	{
		yield return new WaitForSeconds(delaySeconds); // delaySeconds 后执行
		action(obj); // 特定功能
	}
	// 使用例子
	StartCoroutine(
		DelayToInvokeDo(
			delegate(GameObject task) {
				task.SetActive(true);
				task.transform.position = Vector3.zero;
				task.transform.rotation = Quaternion.Euler(Vector3.zero);
				task.doSomethings();
			},
			/*传参*/GameObject.Find("task1"),
			1.5f)/*End 匿名委托*/
		);/*End 协程初始*/
		\end{lstlisting}
		
\chapter{事件}
	\section{必然事件}
		继承自\verb|MonoBehaviour 类|后,自动会\textbf{按序}\textit{提供以下方法}:
		
		\begin{itemize}
			\item \verb|Awake()|:在加载场景时运行,用于在游戏开始前完成变量初始化、以及游戏状态之类的变量。
			\item \verb|Start()|:在第一次启动游戏时执行,用于游戏对象的初始化,在\verb|Awake() |函数之后。
			\item \verb|Update()|:是在每一帧运行时必须执行的函数,用于更新场景和状态。
			\item \verb|FixedUpdate()|:与\verb|Update() |函数相似,但是在固定的物理时间后间隔调用,用于物理状态的更新。
			\item \verb|LateUpdate()|:是在\verb|Update() |函数执行完成后再次被执行的,有点类似收尾的东西。 
		\end{itemize}
			
	\section{碰撞事件}
		U3D 的碰撞检测。具体分为三个部分进行实现,碰撞发生进入时、碰撞发生时和碰撞结束,理论上不能穿透
		
		\begin{itemize}
			\item \verb|OnCollisionEnter(Collision collision)| 当碰撞物体间刚接触时调用此方法
			\item \verb|OnCollisionStay(Collision collision)| 当发生碰撞并保持接触时调用此方法
			\item \verb|OnCollisionExit(Collision collision)| 当不再有碰撞时,既从有到无时调用此函数
		\end{itemize}

	\section{触发器事件}
		类似于 红外线开关门, 有个具体的范围,然后进入该范围时,执行某种动作,离开该范围时执行某种动作。类似于物体于一个透明的物体进行碰撞检测,理论上需要穿透,在U3D 中通过勾选 \verb|Is Trigger| 来确定该物体是可以穿透的。
		
		\begin{itemize}
			\item \verb|OnTriggerEnter() | 当其他碰撞体进入触发器时,执行该方法
			\item \verb|OnTriggerStay() | 当其他碰撞体停留在该触发器中,执行该方法
			\item \verb|OnTriggerExit() | 当碰撞体离开该触发器时,调用该方法
		\end{itemize}
		

\chapter{实体-人物、物体、组件}
		
	\section{实体类} \verb|GameObject 类|,游戏基础对象,用于填充世界。
		\paragraph{复制}
			\verb|Instantiate(GameObject)| 或 \verb|Instantiate(GameObject, position, rotation)|
			
			\begin{itemize}
				\item \verb|GameObject |指生成克隆的\textbf{游戏对象},也可以是\textbf{Prefab 的预制品}
				\item \verb|position |克隆对象的初始位置,类型为\verb|Vector3|
				\item \verb|rotation |克隆对象的初始角度,类型为\verb|Quaternion|
			\end{itemize} 
		
		\paragraph{销毁}
			\verb|Destroy(GameObject xx)- 立即销毁 |或 \verb|Destroy(GameObject xx, Time time)- 几秒后销毁|
			
		\paragraph{可见否}
			通过设置该参数调整该实体是否可以在游戏中显示,具体设置方法为\verb|gameObject.SetActive(true) 为可以显示,false 则隐藏|

	\section{Prefabs -预设体}
		prefabs基础:\url{https://www.cnblogs.com/yuyaonorthroad/p/6107320.html}
		
		动态加载Prefabs:\url{https://blog.csdn.net/linshuhe1/article/details/51355198}
		
		在进行一些功能开发的时候,我们常常将一些\textbf{能够复用的对象}制作成.\textbf{prefab的预设物体},然后将预设体存放到Resources目录之下,使用时再动态加载到场景中并进行实例化。例如:子弹、特效甚至音频等,都能制作成预设体。
		
		\subparagraph{概念} 组件的集合体, 预制物体可以实例化成游戏对象.
		\subparagraph{作用} 可以重复的创建具有相同结构的游戏对象。	
		
		\subsection{预设动态加载到场景}
			\paragraph{预设体资源加载}\verb|->|
			
				假设预设体的位置为下图所示
				\begin{figure}[H]
					\centering
					\includegraphics[scale=0.6]{Prefab-1.png}
					\caption{Prefab 资源位置}
				\end{figure}
			
				\begin{lstlisting}[xleftmargin = .079\textwidth, frame = L]
	//加载预设体资源
	GameObject hp_bar = (GameObject)Resources.Load("Prefabs/HP_Bar");				
				\end{lstlisting}
				
			通过上述操作,实现从资源目录下\textbf{载入}\verb|HP_Bar.prefab|\textbf{预设体},\textbf{用}一个\verb|GameObject|\textbf{对象来存放},此时该预设物体并未真正载入到场景中,因为还未进行实例化操作。
			
			\paragraph{预设体实例化}\verb|->|
			
				实例化使用的是\verb|MonoBehaviour.Instantiate|函数来完成的,\textbf{其实质就是从预设体资源中克隆出一个对象},它\textit{具有与预设体完全相同的属性},并且被加载到当前场景中
				
				完成以上代码之后,在当前场景中会出现一个实例化之后的对象,并且其父节点默认为当场的场景最外层,如下图所示。
					\begin{figure}[H]
						\centering
						\includegraphics[scale=0.8]{Prefab-2.png}
						\caption{Prefab 实例后位置}
					\end{figure}
				
			\paragraph{实例化对象属性设置}\verb|->|
				
				完成上述步骤之后,我们已经可以在场景中看到实例化之后的对象,但是通常情况下我们\textbf{希望}我们的\textbf{对象之间层次感分明},而且这样也方便我们进行对象统一管理,而不是在Hierarchy中看到一大堆并排散乱对象,所以我们\textbf{需要为对象设置名称以及父节点等属性}。
			
				\verb|-->Notice:|常见错误:对\textbf{未初始化}的\verb|hp_bar|进行属性设置,\textbf{设置之后的属性在实例化之后无法生效}。这是\textit{因为我们最后在场景中}\textbf{显示的其实并非实例化前的资源对象},\textbf{而是一个克隆对象},\color{blue}\textit{所以假如希望设置的属性在最后显示出来的对象中生效,我们需要对实例化之后的对象进行设置}。\color{black}
			
			  正确的设置代码如下,可以看到\textbf{实例化对象}已成功挂在到\textbf{父节点Canvas}上,在层次视图效果如下图所示:
			  	\begin{lstlisting}[xleftmargin = .079\textwidth, frame = L]
	GameObject hp_bar = (GameObject)Resources.Load("Prefabs/HP_Bar");
	
	//搜索画布的方法!
	GameObject mUICanvas = GameObject.Find("Canvas");
	hp_bar = Instantiate(hp_bar);
	hp_bar.transform.parent = mUICanvas.transform;
			  	\end{lstlisting}
			  	
			  	\begin{figure}[H]
			  		\centering
			  		\includegraphics[scale=0.8]{Prefab-3.png}
			  		\caption{Prefab对象 设置父子关系}
			  	\end{figure}
		  	
		  		\subparagraph{简化写法}上述实例步骤与属性设置代码可以简化为
		  			\begin{lstlisting}[xleftmargin = .079\textwidth, frame = L]
	GameObject hp_bar = (GameObject)Instantiate(Resources.Load("Prefabs/HP_Bar"));
	GameObject mUICanvas = GameObject.Find("Canvas");
	hp_bar.transform.parent = mUICanvas.transform;	  				
		  			\end{lstlisting}
		  	
		  	\paragraph{预制体添加脚本}
		  		在预制体上不能直接添加脚本,首先需要将其拖入场景,然后再对其操作,这个时候可以添加脚本,添加组件等,在完成这些操作后,在Inspector 选项中选中 Apply,然后删除其在场景中的刚才拖过来的,即可。		
	\section{获取实体上的组件}
		\paragraph{调用方式}\verb|GameObject.GetComponent<Type>().xx = xx;|
			
			\begin{itemize}
				\item \verb|cube1.GetComponent<RigidBody>().mass = 20;| //设置重量
				\item \verb|cube1.GetComponent<BoxCollider>().isTrigger = true;| //\textbf{开启Trigger 穿透方式}检测
				\item \verb|cube2.GetComponent<Test>().enable = false;| //\textbf{禁用Test脚本}
			\end{itemize}
	
	\section{物理作用实体类} \verb|Rigidbody 类|,一种特殊的游戏对象,该类对象可以在物理系统的控制下来运动。
		\paragraph{AddForce()}
			此方法调用时\verb|rigidBody.AddForce(1, 0, 0);|,会施加给刚体一个瞬时力,在力的作用下,会产生一个加速度进行运动。
			
		\paragraph{AddTorque()}
			给刚体添加一个扭矩。
			
		\paragraph{Sleep()}
			使得刚体进入休眠状态,且至少休眠一帧。类似于暂停几帧的意思,这几帧不进行更新、理论位置也不进行更新。
			
		\paragraph{WakeUp()}
			使得刚体从休眠状态唤醒。
		
\chapter{世界变换}
	\section{Transform 类}
		\url{https://blog.csdn.net/yangmeng13930719363/article/details/51460841}
		\subsection{位置}
				transform.position = new Vector3(1, 0, 0);
		\subsection{旋转}	
				transform.Rotate(x, y, z);
				
				transform.eulerAngles = new Vector3(x, y, z);	
		\subsection{缩放}
				transform.localScale(x, y, z); // 基准为1、1、1, 数为缩放因子。	
		\subsection{平移}
				transform.Translate(x, y, z); 
		\subsection{Transform.localPosition}
			\verb|position|是世界坐标中的位置,可以理解为绝对坐标 
			
			\verb|localPosition|是\textbf{相对于父对象的位置},是相对坐标,既父级窗体为原点坐标
		\subsection{注意}
			在变化的过程中需要乘以 Time.deltaTime ,否则会出现大幅不连贯的画面。
	
	\section{摄像机 -Camera}
		\subsection{Clear Flags}
			清除标记。决定屏幕的哪部分将被清除。一般用户使用对台摄像机来描绘不同游戏对象的情况,有3中模式选择:
			\begin{itemize}
				\item \verb|Skybox|:天空盒。默认模式。在屏幕中的空白部分将显示当前摄像机的天空盒。如果当前摄像机没有设置天空盒,会默认用Background色。
				\item \verb|Solid Color|:纯色。选择该模式屏幕上的空白部分将显示当前摄像机的background色。
				\item \verb|Depth only|:仅深度。该模式用于游戏对象不希望被裁剪的情况。
				\item \verb|Dont Clear|:不清除。该模式不清除任何颜色或深度缓存。其结果是,每一帧渲染的结果叠加在下一帧之上。一般与自定义的shader配合使用。
			\end{itemize}
		\subsection{Culling Mask -剔除遮罩} 
			剔除遮罩,选择所要显示的\verb|layer|
		
		\subsection{Projection -透视模式}
			\paragraph{透视}
				摄像机模式
				
			\paragraph{正交}
				前后显示一样,不存在远小近大的样子。
				
		\subsection{Clipping Planes -裁剪模式}
			剪裁平面。摄像机开始渲染与停止渲染之间的距离。
			
		\subsection{Viewport Rect}
			标准视图矩形。用四个数值来控制摄像机的视图将绘制在屏幕的位置和大小,使用的是屏幕坐标系,数值在0~1之间。坐标系原点在左下角。
			
		\subsection{Depth -控制渲染顺序}
			深度。\textbf{用于控制摄像机的渲染顺序},\textbf{较大}值的摄像机将\textbf{被}渲染\textit{在}\textbf{较小}值的摄像机\textbf{之上}。
			
		\subsection{Rendering Path -渲染路径}
			渲染路径。\textbf{用于指定摄像机的渲染方法}。
			
			\verb|Use Player Settings|:使用\verb|Project Settings-->Player|中的设置。
			\verb|Vertex Lit|:\textbf{顶点光照}。摄像机将对所有的游戏对象座位顶点光照对象来渲染。
			\verb|Forward|:\textbf{快速渲染}。摄像机将所有游戏对象将按每种材质一个通道的方式来渲染。
			\verb|Deferred Lighting|:\textbf{延迟光照}。摄像机先对所有游戏对象进行一次无光照渲染,用屏幕空间大小的Buffer保存几何体的深度、法线已经高光强度,生成的Buffer将用于计算光照,同时生成一张新的光照信息Buffer。最后所有的游戏对象会被再次渲染,渲染时叠加光照信息Buffer的内容。
			
		\subsection{Target Texture -目标纹理}
			用于将摄像机视图输出并渲染到屏幕。一般用于制作导航图或者画中画等效果。
			
		\subsection{HDR -高动态光照渲染}
			高动态光照渲染。用于启用摄像机的高动态范围渲染功能。
			
\chapter{键盘鼠标控制}
	\section{普通按键 -keyDown(KeyCode xx)}
		\paragraph{方式一}
		
			\begin{itemize}
				\item 定义按键码:\verb|KeyCode keycode;|
				\item 判断键是否被按下:\verb|if(Input.GetKeyDown(keycode)){}|
				\item 在\verb|Inspirit -> Keycode |指定关联按键
			\end{itemize}
	
		\paragraph{方式二}
			\begin{itemize}
				\item 在\verb|Update| 中更新添加如下代码
				\item \verb|if(Input.GetKeyDown(KeyCode.UpArrow))| 
				\item \verb|KeyCode.xx |包括了键盘所有的按键,常用的AWSD 如下 
					\begin{itemize}
						\item \verb|if (Input.GetKeyDown(KeyCode.S)) |
						\item \verb|if (Input.GetKeyDown(KeyCode.W)) |
					\end{itemize}
			\end{itemize}
		
	\section{根据输入设备 -getAxis()}
		参数分为两类: 
		\paragraph{一、触屏类}
		 	\begin{enumerate}
			 	\item \verb|Mouse X| 鼠标沿屏幕X移动时触发 
			 	\verb|Mouse Y| 鼠标沿屏幕Y移动时触发 
			 	\verb|Mouse ScrollWheel |鼠标滚轮滚动是触发 
		 	\end{enumerate}
			
			\begin{lstlisting}
	float mouseX = Input.GetAxis("Mouse X");
	float mouseY = Input.GetAxis("Mouse Y");
	
	transform.Rotate(Vector3.Up * mouseX * rotateSpeed); // 根据具体需求进行操作
			\end{lstlisting}
		
		\paragraph{二、键盘类}
		 	\begin{enumerate}
		 		\item Vertical 键盘按上或下键时触发 
		 		\item Horizontal 键盘按左或右键时触发
		 	\end{enumerate}
		 	
		 	\begin{lstlisting}
	float horizontal = Input.GetAxis("Horizontal");
	float vertical = Input.GetAxis("Vertical");
	
	Vector3 desPos = (transform.forward * vertical  + transform.right * horizontal) * Time.deltaTime * moveSpeed;
	
	_rigidBody.position += desPos;
		 	\end{lstlisting}
		\textbf{返回值}是一个数,正负代表方向
			
\chapter{时间}
 	\section{Time 类}
 		该类是 U3D 在游戏中获取时间信息的接口类。常用变量如下:
 		
 		\begin{table}[H]
 			\centering
 			\caption{时间变量对照表}
 			\begin{tabular}{m{4cm}|m{10cm}}
 				\toprule
 					变量名 & 意义\\
 				\midrule
 					time & 单位为秒 \\
 					\verb|deltaTime|     & 从上一帧到当前帧消耗的时间 \\
 					fixedTime     & 最近FixedUpdate 的时间,从游戏开始计算 \\
 					\verb|fixedDeltaTime|     & 物理引擎和FixedUpdate 的更新时间间隔 \\
 					timeSceneLevelLoad     & 从当前Scene 开始到目前为止的时间 \\
 					realTimeSinceStartup     & 程序已经运行的时间 \\
 					\verb|frameCount|     & 已经渲染的帧的总数 \\
 				\bottomrule 
 			\end{tabular}
 		\end{table}
 		 		

\chapter{数学}
	\section{Random 类}
		随机数类
		
	\section{Mathf 类}
		数学类
		
\chapter{物理}
	\section{流程}
		\begin{itemize}
			\item \verb|RigidBody |:创建,以完成受力接收。
			\item \verb|Physical Material|:创建,以完成多种力的添加。
			\item \verb|Material |:拖入材质球。
		\end{itemize}

\chapter{光照}
	\section{光照}
		
	\section{烘培}
		\paragraph{简介}
			只有静态场景才能完成烘培(Bake)操作,其目的是在游戏编译阶段完成光照和阴影计算,然后以贴图的形式保存在资源中,以这种手段避免在游戏运行中计算光照而带来的CPU和GPU损耗。
			
			\begin{itemize}
				\item \textbf{如果不烘培}:游戏运行时,这些阴影和反光是由CPU和GPU计算出来的。
				\item \textbf{如果烘焙}:游戏运行时,直接加载在编译阶段完成的光照和阴影贴图,这样就不用再进行计算,节约资源。
			\end{itemize}
		
		\paragraph{流程}
	
	
\chapter{寻路}
	\section{简介}
		NPC 完成自动寻路的功能。
		
	\section{流程}
		\begin{itemize}
			\item 将静态场景调至(Navigation Static)
			\item 烘焙
			\item 添加 \verb|Navigation Mesh Agent| 寻路组件
			\item 在脚本中设置组件的目标地址,添加目标
		\end{itemize}
	
\chapter{UGUI}
	在脚本中使用时记得加上\verb|using UnityEngine.UI |
	
	\url{https://blog.csdn.net/wangmeiqiang/article/category/6364468}
	
	\section{Canvas}
		Canvas画布\textbf{是承载所有UI元素的区域}。\verb|Canvas|实际上\textbf{是一个游戏对象}上\textbf{绑定了Canvas组件}。
		
		\textbf{所有的UI元素}都\textbf{必须是Canvas的子对象}。如果场景中没有画布,那么我们创建任何一个UI元素,都会自动创建画布,并且将新元素置于其下。
		
		在Canvas的\verb|Render Mode|中有三个选择:
			\begin{enumerate}[itemindent = 1em]
				\item Screen Space - Overlay 屏幕最上层,主要是2D效果。
				\item Screen Space - Camera 绑定摄像机,可以实现3D效果。
				\item World Space 世界空间,让UI变成场景中的一个物体。
			\end{enumerate}
	
		\subsection{Screen Space-Overlay -覆盖模式}
			Screen Space-Overlay(屏幕控件-覆盖模式)的\textbf{画布会填满整个屏幕空间},并将画布下面的所有的UI元素置于屏幕的最上层,或者说\textbf{画布的画面永远“覆盖”其他普通的3D画面},\textit{如果屏幕尺寸被改变,画布将自动改变尺寸来匹配屏幕}
			
			Screen Space-Overlay模式的画布有Pixel Perfect和Sort Layer两个参数:
			\begin{enumerate}[itemindent = 1em]
				\item \verb|Pixel Perfect|:只有\verb|RenderMode|为Screen类型时才有的选项。使UI元素像素对应,效果就是\textbf{边缘清晰不模糊}。
				\item \verb|Sort Layer|: \verb|Sort Layer|是UGUI专用的设置,用来指示\textbf{画布的深度}。
			\end{enumerate}
		
		\subsection{Screen Space-Camera -摄像机模式}
			与Screen Space-Overlay模式类似,画布也是\textbf{填满整个屏幕空间},如果屏幕尺寸改变,\textbf{画布也会自动改变尺寸来匹配屏幕}。
			
			不同的是,在该模式下,\textbf{画布会被放置到摄影机前方}。在这种渲染模式下,\textbf{画布看起来 绘制在一个与摄影机固定距离的平面上}。\textit{所有的UI元素都由该摄影机渲染,因此摄影机的设置会影响到UI画面}。在此模式下,UI元素是由\verb|perspective|也就是视角设定的,视角广度由\verb|Filed of View|设置。
			
			\textbf{这种模式可以用来实现在UI上显示3D模型的需求},比如很多MMO游戏中的查看人物装备的界面,可能屏幕的左侧有一个运动的3D人物,左侧是一些UI元素。通过设置Screen Space-Camera模式就可以实现上述的需求,效果如下图所示:
				\begin{figure}[H]
					\centering
					\includegraphics[scale=0.3]{Canva-1.png}
					\includegraphics[scale=0.3]{Canvas.png}
					\caption{摄像机模式-画布}
				\end{figure}
			
			它比Screen Space-Overlay模式的画布多了下面几个参数:
				\begin{enumerate}[itemindent = 1em]
					\item \verb|Render Camera|:渲染摄像机
					\item \verb|Plane Distance|:\textbf{画布距离摄像机的距离}
					\item \verb|Sorting Layer|: Sorting Layer是UGUI专用的设置,\textbf{用来指示画布的深度}。可以通过点击该栏的选项,在下拉菜单中点击“Add Sorting Layer”按钮进入标签和层的设置界面,或者点击导航菜单->edit->Project Settings->Tags and Layers进入该页面。
					\item \verb|Order in Layer|:\textbf{在相同的Sort Layer下的画布显示先后顺序。}数字越高,显示的优先级也就越高。
				\end{enumerate}
			
		\subsection{World Space -世界空间模式}
			World Space即世界空间模式。在此模式下,\textbf{画布被视为与场景中其他普通游戏对象性质相同的类似于一张面片(Plane)的游戏物体}。
			
			画布的尺寸可以\textbf{通过RectTransform设置},所有的UI元素可能位于普通3D物体的前面或者后面显示。\textbf{当UI为场景的一部分时,可以使用这个模式}。
			
				\begin{figure}[H]
					\centering
					\includegraphics[scale=0.3]{Canvas-2.png}
					\includegraphics[scale=0.3]{Canvas-3.png}
					\caption{世界空间 模式- 画布}
				\end{figure}
			
		
		\subsection{使用总结}
			\begin{table}[H]
				\centering
				\caption{渲染模式使用场景说明}
				\begin{tabular}{p{4cm}<{\centering}|c|c|c|p{3cm}<{\centering}}
					\toprule
						渲染模式 & 画布匹配屏幕?& 摄像机? & 像素对应 & 适应\\
					\midrule
						覆盖-overlay模式 & 是	& 不需要	& 可选 & 2D \\
						摄像机-camera模式 & 是	& 需要	 & 可选 & 2D+3D\\
						世界空间-world模式 & 否	& 需要	 & 不可选 & 3D\\
					\bottomrule
				\end{tabular}
			\end{table}
		
		\subsection{Canvas Scalar}
			\url{https://blog.csdn.net/qq168213001/article/details/49744899}
		
		\subsection{Layer}
		
		
	\section{RectTransform}
		\url{https://blog.csdn.net/jk823394954/article/details/53861539}
	
		\url{https://blog.csdn.net/rickshaozhiheng/article/details/51569073}
		
		\url{https://blog.csdn.net/serenahaven/article/details/78826851}
		
		核心看:\url{https://blog.csdn.net/Happy_zailing/article/details/78835482}
		
		\url{http://lib.csdn.net/article/unity3d/36875}
		
		RectTransform继承自Transform, 又增加锚点、中心轴点等信息,\textbf{主要提供一个矩形的位置、尺寸、锚点和中心信息以及操作这些属性的方法},\textit{同时提供多种基于父级RectTransform的缩放形式}。
		
		\subsection{Pivot(中心)}
			Pivot用来指示一个\verb|RectTransform|(或者说是矩形)的中(重)心点。	
	
		\subsection{锚点- 自适应屏幕}
			\url{http://www.bubuko.com/infodetail-2384845.html}
			
			
			锚点(四个)由\verb|两个Vector2|的向量确定,这两个向量确定两个点,归一化坐标分别是\verb|Min|和\verb|Max|,\textbf{再由这两个点确定一个矩形},\textbf{这个矩形的四个顶点就是锚点}。
			
			在\verb|Hierarchy|下新建一个Image,查看其\verb|Inspector|。
			
			\begin{figure}[H]
				\centering
				\includegraphics[scale=1.2]{Anchor.png}
				\caption{Anchor 属性}
			\end{figure}
			
			在Min的x、y值分别小于Max的x、y值时,
			\verb|Min |确定矩形\textbf{左下角}的归一化坐标,\verb|Max |确定矩形\verb|右上角|的归一化坐标。
			
			刚创建的Image,其\verb|Anchor的默认值 |为\verb|Min(0.5,0.5)|和\verb|Max(0.5,0.5)|。也就是说,\verb|Min和Max|重合了,四个锚点合并成一点。锚点在Scene中的表示如下:
			
			\begin{figure}[H]
				\centering
				\includegraphics[scale=0.7]{anchorFirst.png}
				\caption{锚点初始位置}
			\end{figure}
		
			将Min和Max的值分别改为\verb|(0.4,0.4)|和\verb|(0.5,0.5)|。可以看见四个锚点已经分开了。
			
			\begin{figure}[H]
				\centering
				\includegraphics[scale=0.5]{anchorChange.png}
				\caption{min Max位置、确定矩形}
			\end{figure}
			
			
			\subparagraph{需要注意} 在不同的Anchor设置下,控制该RectTransform的变量是不同的。
			
			比如设置成全部居中(默认)时,属性里包含熟悉的用来描述位置的\verb|PosX、PosY和PosZ|,以及用来描述尺寸的\verb|Width和Height|;
			
			切换成全部拉伸时,属性就变成了\verb|Left、Top、Right、Bottom|和\verb|PosZ|,\textbf{前四个属性}用来描述该RectTransform\textbf{分别离父级各边的距离},PosZ用来描述该RectTransform在世界空间的Z坐标
			
			\paragraph{锚点类型}
				\begin{itemize}
					\item 位置类型\verb| 左上角、中心等|
					\item 拉伸类型\verb| 纵向拉伸适配、横向、整体|
				\end{itemize}
		
				\subparagraph{锚点在一块的时候}
					\begin{itemize}[itemindent = 2em]
						\item Anchor 是打在父级窗体上的
						\item Anchor 的位置在父级窗体上的标记方式是按照百分比记录的,单位(百分比)
						\item Anchor 的\verb|Min(RectTransform.anchorMin)  Max(RectTransform.anchorMax)|的信息保持一致
						\item 子物件的 坐标系 为纵向Y,横向X, 并且以\verb|Anchor| 为原点, 自身坐标用中心轴点\verb|Pivot| 表示						
						\item 子物件的 Pivot 与 Anchor  位置始终保持不变,单位(像素)
					\end{itemize}
				
				\subparagraph{锚点单向(横或者纵)分开的时候}
					\begin{itemize}[itemindent = 2em]
						\item 分开的部分(拉伸方向)与父级窗体保持一致变化,单位(百分比)
						\item 与相对方向则绝对保持,单位(像素)
					\end{itemize}
				
				\subparagraph{锚点双向分开的时候}
					\begin{itemize}[itemindent = 2em]
						\item 双向 都与 父级窗体 保持一致的变化,单位(百分比)
						\item 上-top、下-bottom、左、右边距绝对保持,单位(像素)
					\end{itemize}
		
			\paragraph{anchorMax、anchorMin}
				\verb|anchorMin.x|表示锚点在\verb|x|轴的起始点位置,\verb|anchorMax.x|表示锚点在x轴的终点位置,取值\verb|0~1|,表示\textbf{百分比值},该值乘以父窗口的\verb|width|值就是实际锚点相对于父窗口x轴的位置。y轴与x轴同理。
		
					\begin{figure}[H]
						\centering
						\includegraphics[scale=0.8]{Anchors-3.png}
						\caption{Anchor.Min 与 Anchor.Max}
					\end{figure}
		
				这个值确定了锚点相对于父窗口的位置,是\textbf{真正决定锚点位置的值}
				
			\paragraph{offsetMax 和 offsetMin 属性}
				\subparagraph{锚点 分开时}
					在锚点分开的状态下: 锚点其实是四个钉子,分为左上,左下,右下及右上四个,每个空间在UI模型中都是一个矩形,也有左上,左下,右下及右上四个顶点,那么锚点的每个钉子可以关联一个点,即左上————左上;左下————左下;右下————右下;右上————右上。这样进行绑定。
					
					\verb|offsetMax |是RectTransform\textbf{右上角相对于右上Anchor的距离};
			
					\verb|offsetMin |是RectTransform\textbf{左下角相对于左下Anchor的距离}。
			
					\verb|offset |可以认为是以像素为单位。
					
					\begin{figure}[H]
						\centering
						\includegraphics[scale=0.7]{Anchors-1.png}
						\caption{锚点在一起时  Offset 求取向量示例}
						\label{锚点分开时 Offset 求取向量示例}
					\end{figure}
			
				\subparagraph{锚点 在一处时}
					锚点offset 计算如下:
					
					\begin{figure}[H]
						\centering
						\includegraphics[scale=0.7]{Anchors-2.png}
						\caption{锚点分开时  Offset 求取向量示例}
					\end{figure}
			
				\subparagraph{求取}
					首先计算锚点的每个钉子到其对应的顶点矢量值,分别记作\verb|v0|,\verb|v1|,\verb|v2|,\verb|v3|, 入上图。
					
					然后比较四个向量的\verb|x|值,将\verb|x|的最大值赋给\verb|offsetMax.x|,将\verb|x|的最小值赋给\verb|offsetMin.x|;\verb|y|的值同理。
			
			\paragraph{anchoredPosition}
				\subparagraph{锚点 在一处时}
					anchorPosition 就是 \textbf{从锚点}到本物体的\textbf{轴心}(Pivot)的\textbf{向量值}.
				
				\subparagraph{锚点 分开时}
						
		\subsection{sizeDelta}
		
			\verb|sizeDelta|是\verb|offsetMax-offsetMin|的结果。在\textit{锚点全部重合的情况下},它的值就是面板上的\verb|(Width,Height)|。
			
			\textit{在锚点完全不重合}的情况下,它是相对于父矩形的尺寸。
			
			一个常见的错误是,当RectTransform的锚点\textbf{并非全部重合时},使用sizeDelta作为这个RectTransform的尺寸。此时拿到的结果一般来说并非预期的结果。
			
		\subsection{RectTransform.rect}	
			RectTransform.rect 的各值如图所示。
			
			\begin{figure}[H]
				\centering
				\includegraphics[scale=0.8]{RectTransform-rect.png}
				\caption{RectTransform rect 属性}
			\end{figure}	
		
		\subsection{示例}
			\begin{lstlisting}
	GameObject webText = new GameObject("webText");
	webText.AddComponent<UnityEngine.UI.Text>();
	webText.GetComponent<UnityEngine.UI.Text>().text = "";
	webText.GetComponent<RectTransform>().anchorMin = new Vector2(0, 0);
	webText.GetComponent<RectTransform>().anchorMax = new Vector2(1, 1);
	webText.GetComponent<RectTransform>().sizeDelta = new Vector2(0, 0);
	webText.GetComponent<RectTransform>().anchoredPosition = new Vector2(0, 0);
	webText.transform.localPosition = new Vector3(0,0,0);
	webText.transform.SetParent(webObj.transform, false);
			\end{lstlisting}
			
		\subsection{FramDebug}
			查看渲染的先后顺序
			
			\verb|windows->FrameDebug |	
	\section{按钮}
		\subsection{原始Button}
			
		\subsection{Image等 -添加button 组件}
			\begin{itemize}
				\item \verb|create -> UI -> Image|
				\item \verb|Inspirit -> Add Component -> button|
			\end{itemize}	
		
		\subsection{添加事件处理脚本}
			\begin{itemize}
				\item 书写脚本并添加到Button gameObject 上
				\item 如果是Button 组件的话直接在button 组件上添加,如果是Image 则添加button 组件后再添加
				\item  添加脚本对象到\verb|onClick() |部分:\verb|+ -> gameObject 拖进来  -> 选择脚本中的具体函数|
			\end{itemize}
		
	\section{文本- Text}
		\subsection{添加文字阴影 -shadow 组件}
			\verb|addComponent -> shadow|
			
		\subsection{添加文子边框 -outline 组件}
			\verb|addComponent -> outline|
			
	\section{图片- ImageView}
		
	\section{选中标记- Toggle}
		\paragraph{Toggle 基本}
		
		\paragraph{Toggle Group}
			\subparagraph{选项栏设定}
				将panel  拖入 toggle 中的\verb|value changed|部分
				
			\subparagraph{预设}
				确定默认打开哪个panel,然后将其\verb|IsOn| 勾选,其余取消勾选	
				
	\section{滚动区域、滚动条}
	
	
	\section{其他工具条}
	
	\section{布局- Layout}
		\begin{itemize}
			\item 具体页面下创建空物体 \verb| GameObject|
			\item 其次在\verb|GameObject 下|添加组件 \verb|-> grid layout group|
			\item 最后在这个\verb|GameObject |下创建出各种Image 组件,然后这些组件将会以\verb|grid layout| 的布局进行自动调整
		\end{itemize}
	
		\subsection{grid layout group}
			\begin{itemize}
				\item  调整\verb|cell size| 进行调整子物件的大小
				\item  \verb|cell size| 的改变只影响子组件的第一层,既最下面一层
			\end{itemize}
		
		\subsection{horizontal layout group}
		
		\subsection{vertical layout group}
		
		
\chapter{着色器渲染}

		    
\chapter{跨平台发布apk}
	\section{流程}
		\begin{itemize}
			\item 安装 JavaSDK、Android Studio 并在SDK manager 里添加对应的API包
			\item 在unity 中的\verb|edit |选项下的\verb|preferences|, 并选中\verb|External Tools| 选项,配置\verb|JDK |和\verb|Android SDK| 安装位置。
			\item 在unity 中的\verb|File -> Build Settings|中,添加需要添加的场景,并选择对应的平台(Android, IOS)等
			\item 在unity 中的\verb|Build Settings |中的\verb|Player Settings |设置以下几个重要内容。
				\begin{enumerate}
					\item \verb|Company Name |
					\item \verb|Product Name |
					\item \verb|Default Icon |:192$\times$192
					\item \verb|Default Orientation |
					\item \verb|Other Settings -> Identification |:修改为\verb|com.netease(Or Other).TestName(Or Other)|
				\end{enumerate}
		\end{itemize}	
	
	\section{Apk 安装常见错误}
		\url{http://mumu.163.com/2017/03/30/25905_680657.html}


\chapter{调试技巧}
	\section{以父类为基点}
		在Inspector 中查看是否存在父类脚本\verb|[SerializedField]| 的变量,这样方便对空间进行查找,并且添加新的\textbf{控制}			
	
	
		    
\end{document} 
 		    